% \documentclass[12pt,screen]{talk}
\documentclass[12pt,slides]{talk}
% \documentclass[12pt,notes]{talk}

\usepackage{sidebars}


\title{My Test Talk}
\author[John Doe]{John Doe\thanks{Some institute.}}


\begin{document}

\begin{slide}[plain]{}
  \maketitle
\end{slide}

\begin{slide}[outline]{My Table of Contents}
  \tableofcontents
\end{slide}

\section{First Section}

\begin{slide}{My Title}
  Try clicking on the section titles in the sidebar.\footnote{This is a
  footnote.}
\end{slide}
\begin{notes}
  Some notes about this slide.
\end{notes}

\subsection{First Subsection}
\begin{slide}{Another Title}
  More stuff.
\end{slide}
\begin{slide}{Yet Another Title}
  Even more stuff.
\end{slide}

\subsection{Second Subsection}
\begin{slide}{First Slide in Second Subsection}
  Blah blah bla...
\end{slide}

\section{Second Section}

\begin{slide}{A Slide in Section Two}
  ...you get the idea.
\end{slide}

\subsection{Subsection 2.1}
\begin{multislide}{3}{Multislide Example}
  A few bullet points:
  \begin{itemize}
  \item Bullet 1
  \fromslide{2}{\item Bullet 2}
  {\onlyslide*{3}{\color{red}}\item Bullet 3}
  \end{itemize}
  \onlyslide*{1}{You can only see bullets 1 and 3.}%
  \onlyslide*{2}{You can see all bullets.}
  \onlyslide*{3}{Now bullet 3 is red.}
\end{multislide}

\begin{multislide}{2}{Another Multislide}
  ding \fromslide{2}{dong.}
\end{multislide}

\end{document}
